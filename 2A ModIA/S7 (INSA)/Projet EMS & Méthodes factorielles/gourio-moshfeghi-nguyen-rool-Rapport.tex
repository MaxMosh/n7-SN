% Options for packages loaded elsewhere
\PassOptionsToPackage{unicode}{hyperref}
\PassOptionsToPackage{hyphens}{url}
%
\documentclass[
]{article}
\usepackage{amsmath,amssymb}
\usepackage{iftex}
\ifPDFTeX
  \usepackage[T1]{fontenc}
  \usepackage[utf8]{inputenc}
  \usepackage{textcomp} % provide euro and other symbols
\else % if luatex or xetex
  \usepackage{unicode-math} % this also loads fontspec
  \defaultfontfeatures{Scale=MatchLowercase}
  \defaultfontfeatures[\rmfamily]{Ligatures=TeX,Scale=1}
\fi
\usepackage{lmodern}
\ifPDFTeX\else
  % xetex/luatex font selection
\fi
% Use upquote if available, for straight quotes in verbatim environments
\IfFileExists{upquote.sty}{\usepackage{upquote}}{}
\IfFileExists{microtype.sty}{% use microtype if available
  \usepackage[]{microtype}
  \UseMicrotypeSet[protrusion]{basicmath} % disable protrusion for tt fonts
}{}
\makeatletter
\@ifundefined{KOMAClassName}{% if non-KOMA class
  \IfFileExists{parskip.sty}{%
    \usepackage{parskip}
  }{% else
    \setlength{\parindent}{0pt}
    \setlength{\parskip}{6pt plus 2pt minus 1pt}}
}{% if KOMA class
  \KOMAoptions{parskip=half}}
\makeatother
\usepackage{xcolor}
\usepackage[margin=1in]{geometry}
\usepackage{color}
\usepackage{fancyvrb}
\newcommand{\VerbBar}{|}
\newcommand{\VERB}{\Verb[commandchars=\\\{\}]}
\DefineVerbatimEnvironment{Highlighting}{Verbatim}{commandchars=\\\{\}}
% Add ',fontsize=\small' for more characters per line
\usepackage{framed}
\definecolor{shadecolor}{RGB}{248,248,248}
\newenvironment{Shaded}{\begin{snugshade}}{\end{snugshade}}
\newcommand{\AlertTok}[1]{\textcolor[rgb]{0.94,0.16,0.16}{#1}}
\newcommand{\AnnotationTok}[1]{\textcolor[rgb]{0.56,0.35,0.01}{\textbf{\textit{#1}}}}
\newcommand{\AttributeTok}[1]{\textcolor[rgb]{0.13,0.29,0.53}{#1}}
\newcommand{\BaseNTok}[1]{\textcolor[rgb]{0.00,0.00,0.81}{#1}}
\newcommand{\BuiltInTok}[1]{#1}
\newcommand{\CharTok}[1]{\textcolor[rgb]{0.31,0.60,0.02}{#1}}
\newcommand{\CommentTok}[1]{\textcolor[rgb]{0.56,0.35,0.01}{\textit{#1}}}
\newcommand{\CommentVarTok}[1]{\textcolor[rgb]{0.56,0.35,0.01}{\textbf{\textit{#1}}}}
\newcommand{\ConstantTok}[1]{\textcolor[rgb]{0.56,0.35,0.01}{#1}}
\newcommand{\ControlFlowTok}[1]{\textcolor[rgb]{0.13,0.29,0.53}{\textbf{#1}}}
\newcommand{\DataTypeTok}[1]{\textcolor[rgb]{0.13,0.29,0.53}{#1}}
\newcommand{\DecValTok}[1]{\textcolor[rgb]{0.00,0.00,0.81}{#1}}
\newcommand{\DocumentationTok}[1]{\textcolor[rgb]{0.56,0.35,0.01}{\textbf{\textit{#1}}}}
\newcommand{\ErrorTok}[1]{\textcolor[rgb]{0.64,0.00,0.00}{\textbf{#1}}}
\newcommand{\ExtensionTok}[1]{#1}
\newcommand{\FloatTok}[1]{\textcolor[rgb]{0.00,0.00,0.81}{#1}}
\newcommand{\FunctionTok}[1]{\textcolor[rgb]{0.13,0.29,0.53}{\textbf{#1}}}
\newcommand{\ImportTok}[1]{#1}
\newcommand{\InformationTok}[1]{\textcolor[rgb]{0.56,0.35,0.01}{\textbf{\textit{#1}}}}
\newcommand{\KeywordTok}[1]{\textcolor[rgb]{0.13,0.29,0.53}{\textbf{#1}}}
\newcommand{\NormalTok}[1]{#1}
\newcommand{\OperatorTok}[1]{\textcolor[rgb]{0.81,0.36,0.00}{\textbf{#1}}}
\newcommand{\OtherTok}[1]{\textcolor[rgb]{0.56,0.35,0.01}{#1}}
\newcommand{\PreprocessorTok}[1]{\textcolor[rgb]{0.56,0.35,0.01}{\textit{#1}}}
\newcommand{\RegionMarkerTok}[1]{#1}
\newcommand{\SpecialCharTok}[1]{\textcolor[rgb]{0.81,0.36,0.00}{\textbf{#1}}}
\newcommand{\SpecialStringTok}[1]{\textcolor[rgb]{0.31,0.60,0.02}{#1}}
\newcommand{\StringTok}[1]{\textcolor[rgb]{0.31,0.60,0.02}{#1}}
\newcommand{\VariableTok}[1]{\textcolor[rgb]{0.00,0.00,0.00}{#1}}
\newcommand{\VerbatimStringTok}[1]{\textcolor[rgb]{0.31,0.60,0.02}{#1}}
\newcommand{\WarningTok}[1]{\textcolor[rgb]{0.56,0.35,0.01}{\textbf{\textit{#1}}}}
\usepackage{graphicx}
\makeatletter
\def\maxwidth{\ifdim\Gin@nat@width>\linewidth\linewidth\else\Gin@nat@width\fi}
\def\maxheight{\ifdim\Gin@nat@height>\textheight\textheight\else\Gin@nat@height\fi}
\makeatother
% Scale images if necessary, so that they will not overflow the page
% margins by default, and it is still possible to overwrite the defaults
% using explicit options in \includegraphics[width, height, ...]{}
\setkeys{Gin}{width=\maxwidth,height=\maxheight,keepaspectratio}
% Set default figure placement to htbp
\makeatletter
\def\fps@figure{htbp}
\makeatother
\setlength{\emergencystretch}{3em} % prevent overfull lines
\providecommand{\tightlist}{%
  \setlength{\itemsep}{0pt}\setlength{\parskip}{0pt}}
\setcounter{secnumdepth}{-\maxdimen} % remove section numbering
\ifLuaTeX
  \usepackage{selnolig}  % disable illegal ligatures
\fi
\IfFileExists{bookmark.sty}{\usepackage{bookmark}}{\usepackage{hyperref}}
\IfFileExists{xurl.sty}{\usepackage{xurl}}{} % add URL line breaks if available
\urlstyle{same}
\hypersetup{
  pdftitle={Projet d'étude : Analyse de données \& Éléments de modélisation statistique},
  hidelinks,
  pdfcreator={LaTeX via pandoc}}

\title{Projet d'étude : Analyse de données \& Éléments de modélisation
statistique}
\author{}
\date{\vspace{-2.5em}2023-09-29}

\begin{document}
\maketitle

\begin{verbatim}
## corrplot 0.92 loaded
\end{verbatim}

\hypertarget{statistiques-descriptives}{%
\section{Statistiques descriptives}\label{statistiques-descriptives}}

Le jeu de données comprend \(984\) mesures des émissions de polluants
atmosphériques tous secteurs d'activités confondues des EPCI
(Etablissements Publics de Coopération Intercommunale) de la région
Occitanie de 2014 à 2019.

Chaque mesure est décrite par les variables qualitatives suivantes : +
\emph{lib\_epci} : son nom + \emph{code\_epci} : son code
d'identification + \emph{nomdepart} : son (ses) département(s)
d'appartenance + \emph{TypeEPCI} : CC (communauté de commune), CA
(communauté d'agglomération), Métrople et CU (communauté urbaine) +
\emph{annee\_inv} : l'année de mesure

Et par les variables quantitatives suivantes : + \emph{nox\_kg} : oxyde
d'azote en kg + \emph{so2\_kg} : oxyde de soufre en kg + \emph{pm10\_kg}
: particules en suspension dans l'air de diamètre inférieur à 10 µm +
\emph{pm25\_kg} : particules en suspension dans l'air de diamètre
inférieur à 2.5 µm + \emph{co\_kg} : monoxyde de carbone +
\emph{c6h6\_kg} : benzène + \emph{nh3\_kg} : ammoniac +
\emph{ges\_teqco2} : gaz à effet de serre + \emph{ch4\_t} : méthane +
\emph{co2\_t} : dioxyde de carbone + \emph{n2o\_t} : protoxyde d'azote +
\emph{latit} : sa latitude + \emph{longit} : sa longitude

Dans la suite de ce rapport, nous utilisons la notation \(\Delta\) pour
faire référence à des plots qui sont disponibles dans le Rmd mais que
nous avons décidé de ne pas inclure dans le rapport.

\begin{verbatim}
##   code_epci                            lib_epci annee_inv     nox_kg    so2_kg
## 1 200006930                   CC du Haut Allier      2019   65633.66  3866.599
## 2 200017341               CC Lodévois et Larzac      2019  310288.20  8083.028
## 3 200022986          CC du Grand Pic Saint-Loup      2019  337655.55  9373.106
## 4 200023620       CC de la Gascogne Toulousaine      2019  298100.30  4091.852
## 5 200023737                  CA du Grand Cahors      2019  447186.53 13650.148
## 6 200027183 CU Perpignan Méditerranée Métropole      2019 2110865.51 57993.192
##     pm10_kg   pm25_kg     co_kg   c6h6_kg    nh3_kg ges_teqco2   ch4_t
## 1  15728.87  10975.55  173194.3  2319.199 133686.18   43995.12 617.104
## 2  50929.20  38591.71  593036.6  8349.081 114533.40  127777.47 436.445
## 3 143623.67  82143.61 1275976.8 18806.497  71177.45  161136.84 251.623
## 4 126735.60  63331.88  780230.6 12250.430 244266.82  116802.18 275.749
## 5 143525.58 111854.31 1386798.2 21346.289 130426.82  216301.79 447.677
## 6 506888.15 353513.88 5270166.1 78510.229 111010.72 1057760.61 398.561
##       co2_t  n2o_t TypeEPCI           nomdepart Ardèche Ariège Aude Aveyron
## 1  17831.59 17.114       CC              Lozère       0      0    0       0
## 2  93016.70 19.755       CC             Hérault       0      0    0       0
## 3 125004.72 17.606       CC             Hérault       0      0    0       0
## 4  79458.03 50.976       CC  Gers,Haute-Garonne       0      0    0       0
## 5 161747.97 24.640       CA                 Lot       0      0    0       0
## 6 806673.97 59.760       CU Pyrénées-Orientales       0      0    0       0
##   Gard Haute.Garonne Gers Hérault Landes Lot Lot.et.Garonne Lozère
## 1    0             0    0       0      0   0              0      1
## 2    0             0    0       1      0   0              0      0
## 3    0             0    0       1      0   0              0      0
## 4    0             1    1       0      0   0              0      0
## 5    0             0    0       0      0   1              0      0
## 6    0             0    0       0      0   0              0      0
##   Pyrénées.Atlantiques Hautes.Pyrénées Pyrénées.Orientales Tarn Tarn.et.Garonne
## 1                    0               0                   0    0               0
## 2                    0               0                   0    0               0
## 3                    0               0                   0    0               0
## 4                    0               0                   0    0               0
## 5                    0               0                   0    0               0
## 6                    0               0                   1    0               0
##   Vaucluse    latit   longit
## 1        0 44.73240 3.769267
## 2        0 43.79291 3.372977
## 3        0 43.77692 3.794306
## 4        0 43.59144 1.066752
## 5        0 44.48480 1.447761
## 6        0 42.75200 2.851604
\end{verbatim}

\hypertarget{analyse-unidimentionnelle}{%
\subsection{Analyse unidimentionnelle}\label{analyse-unidimentionnelle}}

\hypertarget{analyse-des-variables-quantitatives}{%
\subsubsection{Analyse des variables
quantitatives}\label{analyse-des-variables-quantitatives}}

Nous observons avec ce boxplot que nos variables n'ont pas les mêmes
ordres de grandeur, cela va biaiser nos analyses de variances. Nous
appliquons donc une transformation logarithmique aux valeurs
quantitatives.

\begin{verbatim}
## Using  as id variables
## Using  as id variables
\end{verbatim}

\includegraphics{gourio-moshfeghi-nguyen-rool-Rapport_files/figure-latex/unnamed-chunk-8-1.pdf}
Nous remarquons que la transformation logarithme suffit à mieux
visualiser nos données. Nous conserverons ce jeu de données transformé
par la suite. Nous stockons le dataset avec les variables quantitatives
transformées dans la variable \(Datalog\).

Nous affichons maintenant la fonction de répartition de la concentration
d'oxyde d'azote.

\((\Delta)\) D'autres graphes similaires sont disponibles dans le Rmd.

à enlever surement
\includegraphics{gourio-moshfeghi-nguyen-rool-Rapport_files/figure-latex/unnamed-chunk-9-1.pdf}

Les données sont uniformément réparties. Nous utiliserons donc ce
dataset pour la suite de notre étude.

boucle for to do

\hypertarget{analyse-des-variables-qualitatives}{%
\subsubsection{Analyse des variables
qualitatives}\label{analyse-des-variables-qualitatives}}

Nous nous intéressons maintenant aux variables qualitatives :
\(TypeEPCI\) \& \(annee_inv\). \((\Delta)\) Nous remarquons que nous
avons le même nombre de prises de mesure pour chaque année, donc aucunes
données ne semblent manquer.

Nous affichons la répartition des mesures en fonction de \(TypeEPCI\).
Nous remarquons que les catégories CA, CU et Metropole ne comprennent
pas beaucoup de valeurs par rapport au type CC. Nous choissisons donc de
combiner les trois catégories pour avoir un nombre suffisant de valeurs
pour chacunes de nos modalités.

changer titre

\includegraphics{gourio-moshfeghi-nguyen-rool-Rapport_files/figure-latex/unnamed-chunk-17-1.pdf}

\hypertarget{analyse-bidementionnelle}{%
\subsection{Analyse bidementionnelle}\label{analyse-bidementionnelle}}

En affichant la matrice de corrélation des variables de notre dataset,
nous observons que la plupart des polluants sont corrélés positivement
les uns avec les autres à l'exception de \(nh3\_kg\), \(ch4\_t\) et
\(n2o\_t\) qui ne semblent être corrélés à aucune autre variable.

\begin{Shaded}
\begin{Highlighting}[]
\FunctionTok{corrplot}\NormalTok{(}\FunctionTok{cor}\NormalTok{(Datalog\_[}\DecValTok{4}\SpecialCharTok{:}\DecValTok{14}\NormalTok{]), }\AttributeTok{method=}\StringTok{"ellipse"}\NormalTok{)}
\end{Highlighting}
\end{Shaded}

\includegraphics{gourio-moshfeghi-nguyen-rool-Rapport_files/figure-latex/unnamed-chunk-18-1.pdf}

Nous constatons avec le boxplot ci-dessous qu'il y a un lien entre le
type EPCI et les différents polluants. En effet les boxplot en fonction
des types sont à des niveaux différents. Rajouter titre ?

\begin{Shaded}
\begin{Highlighting}[]
\FunctionTok{ggplot}\NormalTok{(Datalog\_, }\FunctionTok{aes}\NormalTok{(}\AttributeTok{x =}\NormalTok{ TypeEPCI, }\AttributeTok{y =}\NormalTok{ pm10\_kg)) }\SpecialCharTok{+} \FunctionTok{geom\_boxplot}\NormalTok{()}
\end{Highlighting}
\end{Shaded}

\includegraphics{gourio-moshfeghi-nguyen-rool-Rapport_files/figure-latex/unnamed-chunk-19-1.pdf}

\hypertarget{visualisation-des-individus}{%
\section{Visualisation des
individus}\label{visualisation-des-individus}}

Nous réalisons une ACP pour visualiser nos données dans un espace de
dimension inférieure.

Nous remarquons avec les graphes ci-dessous que les deux premières
dimensions constituent 90\% de la variance. Nous allons poursuivre les
analyses par ces deux dimensions.

\includegraphics{gourio-moshfeghi-nguyen-rool-Rapport_files/figure-latex/unnamed-chunk-21-1.pdf}
\includegraphics{gourio-moshfeghi-nguyen-rool-Rapport_files/figure-latex/unnamed-chunk-21-2.pdf}

Nous affichons le cercle des corrélations. La première dimension décrit
toutes les variables sauf \(ch4\_t\), \(nh3\_kg\) et \(n2o\_t\) qui
décrivent la deuxième dimension.

\includegraphics{gourio-moshfeghi-nguyen-rool-Rapport_files/figure-latex/unnamed-chunk-22-1.pdf}
Nous remarquons que les 3 polluants qui ne sont pas corrélés avec les
autres polluants (cf corrplot) composent la deuxième dimension de l'ACP.

Nous affichons également le graphe des individus colorés en fonction du
Type EPCI. Nous remarquons que les types CA/CU/Metropole ont tendance à
avoir en plus grandes quantité les polluants de la première dimension.
Nous remarquons aussi deux groupements extrêmes. Un groupement très peu
pollué et un autre plus pollué. Après identification, il s'agit de
Toulouse Métropole et de CC Pays de Nay. Nous décidons d'enlever ces
deux groupements car ils ne représentent pas la tendance général ; ce
sont des outliers qui créeront une grande disparité.

\includegraphics{gourio-moshfeghi-nguyen-rool-Rapport_files/figure-latex/unnamed-chunk-23-1.pdf}

Remarque : Nous constatons, avec le graphique ci dessous, que chaque
regroupement corresponds à une ville grâce à l'habillage par année.
Cette visualisation nous a permis d'éliminer les outliers.

\includegraphics{gourio-moshfeghi-nguyen-rool-Rapport_files/figure-latex/unnamed-chunk-24-1.pdf}

\((\Delta)\) Nous affichons aussi cette représentation sans le
groupement des \(TypeEPCI\). Cela confirme que nous avons un classement
de la pollution en fonction du Type EPCI. Si nous alons du plus au moins
pollué, nous avons : Metropole, CU, CA puis CC.

\(\Delta\) Nous réaffichons les mêmes graphes sans les deux outliers.

\hypertarget{clustering-pas-ruxe9diguxe9}{%
\section{Clustering (pas rédigé)}\label{clustering-pas-ruxe9diguxe9}}

\begin{Shaded}
\begin{Highlighting}[]
\NormalTok{reskmeans}\OtherTok{\textless{}{-}}\FunctionTok{kmeans}\NormalTok{((Data\_final[,}\DecValTok{4}\SpecialCharTok{:}\DecValTok{14}\NormalTok{]),}\AttributeTok{centers=}\DecValTok{5}\NormalTok{)}
\FunctionTok{table}\NormalTok{(reskmeans}\SpecialCharTok{$}\NormalTok{cluster)}
\end{Highlighting}
\end{Shaded}

\begin{verbatim}
## 
##   1   2   3   4   5 
## 170 188 121 296 197
\end{verbatim}

\begin{Shaded}
\begin{Highlighting}[]
\FunctionTok{fviz\_cluster}\NormalTok{(reskmeans,}\AttributeTok{data=}\NormalTok{Data\_final[,}\DecValTok{4}\SpecialCharTok{:}\DecValTok{14}\NormalTok{],}\AttributeTok{ellipse.type=}\StringTok{"norm"}\NormalTok{,}\AttributeTok{labelsize=}\DecValTok{8}\NormalTok{,}\AttributeTok{geom=}\FunctionTok{c}\NormalTok{(}\StringTok{"point"}\NormalTok{), }\AttributeTok{axes =} \FunctionTok{c}\NormalTok{(}\DecValTok{1}\NormalTok{,}\DecValTok{2}\NormalTok{))}\SpecialCharTok{+}\FunctionTok{ggtitle}\NormalTok{(}\StringTok{""}\NormalTok{)}
\end{Highlighting}
\end{Shaded}

\includegraphics{gourio-moshfeghi-nguyen-rool-Rapport_files/figure-latex/unnamed-chunk-28-1.pdf}

\begin{Shaded}
\begin{Highlighting}[]
\NormalTok{reskmeans}\OtherTok{\textless{}{-}}\FunctionTok{kmeans}\NormalTok{((Datalog[,}\DecValTok{4}\SpecialCharTok{:}\DecValTok{14}\NormalTok{]),}\AttributeTok{centers=}\DecValTok{7}\NormalTok{)}
\FunctionTok{table}\NormalTok{(reskmeans}\SpecialCharTok{$}\NormalTok{cluster)}
\end{Highlighting}
\end{Shaded}

\begin{verbatim}
## 
##   1   2   3   4   5   6   7 
## 206  99 128  54 108 170 219
\end{verbatim}

\begin{Shaded}
\begin{Highlighting}[]
\FunctionTok{fviz\_cluster}\NormalTok{(reskmeans,}\AttributeTok{data=}\NormalTok{Datalog[,}\DecValTok{4}\SpecialCharTok{:}\DecValTok{14}\NormalTok{],}\AttributeTok{ellipse.type=}\StringTok{"norm"}\NormalTok{,}\AttributeTok{labelsize=}\DecValTok{8}\NormalTok{,}\AttributeTok{geom=}\FunctionTok{c}\NormalTok{(}\StringTok{"point"}\NormalTok{), }\AttributeTok{axes =} \FunctionTok{c}\NormalTok{(}\DecValTok{1}\NormalTok{,}\DecValTok{2}\NormalTok{))}\SpecialCharTok{+}\FunctionTok{ggtitle}\NormalTok{(}\StringTok{""}\NormalTok{)}
\end{Highlighting}
\end{Shaded}

\includegraphics{gourio-moshfeghi-nguyen-rool-Rapport_files/figure-latex/unnamed-chunk-29-1.pdf}

\begin{Shaded}
\begin{Highlighting}[]
\CommentTok{\# A completer}
\NormalTok{Kmax}\OtherTok{\textless{}{-}}\DecValTok{15}
\NormalTok{reskmeanscl}\OtherTok{\textless{}{-}}\FunctionTok{matrix}\NormalTok{(}\DecValTok{0}\NormalTok{,}\AttributeTok{nrow=}\FunctionTok{nrow}\NormalTok{(Data\_final),}\AttributeTok{ncol=}\NormalTok{Kmax}\DecValTok{{-}1}\NormalTok{)}
\NormalTok{Iintra}\OtherTok{\textless{}{-}}\ConstantTok{NULL}
\ControlFlowTok{for}\NormalTok{ (k }\ControlFlowTok{in} \DecValTok{2}\SpecialCharTok{:}\NormalTok{Kmax)\{}
\NormalTok{  resaux}\OtherTok{\textless{}{-}}\FunctionTok{kmeans}\NormalTok{(Data\_final[,}\DecValTok{4}\SpecialCharTok{:}\DecValTok{14}\NormalTok{],}\AttributeTok{centers=}\NormalTok{k)}
\NormalTok{  reskmeanscl[,k}\DecValTok{{-}1}\NormalTok{]}\OtherTok{\textless{}{-}}\NormalTok{resaux}\SpecialCharTok{$}\NormalTok{cluster}
\NormalTok{  Iintra}\OtherTok{\textless{}{-}}\FunctionTok{c}\NormalTok{(Iintra,resaux}\SpecialCharTok{$}\NormalTok{tot.withinss)}
\NormalTok{\}}

\NormalTok{df}\OtherTok{\textless{}{-}}\FunctionTok{data.frame}\NormalTok{(}\AttributeTok{K=}\DecValTok{2}\SpecialCharTok{:}\DecValTok{15}\NormalTok{,}\AttributeTok{Iintra=}\NormalTok{Iintra)}
\FunctionTok{ggplot}\NormalTok{(df,}\FunctionTok{aes}\NormalTok{(}\AttributeTok{x=}\NormalTok{K,}\AttributeTok{y=}\NormalTok{Iintra))}\SpecialCharTok{+}\FunctionTok{geom\_line}\NormalTok{()}\SpecialCharTok{+}\FunctionTok{geom\_point}\NormalTok{()}\SpecialCharTok{+}\FunctionTok{xlab}\NormalTok{(}\StringTok{"Nombre de classes"}\NormalTok{)}\SpecialCharTok{+}\FunctionTok{ylab}\NormalTok{(}\StringTok{"Inertie intraclasse"}\NormalTok{)}
\end{Highlighting}
\end{Shaded}

\begin{Shaded}
\begin{Highlighting}[]
\CommentTok{\# A completer}
\NormalTok{Kmax}\OtherTok{\textless{}{-}}\DecValTok{15}
\NormalTok{reskmeanscl}\OtherTok{\textless{}{-}}\FunctionTok{matrix}\NormalTok{(}\DecValTok{0}\NormalTok{,}\AttributeTok{nrow=}\FunctionTok{nrow}\NormalTok{(Datalog),}\AttributeTok{ncol=}\NormalTok{Kmax}\DecValTok{{-}1}\NormalTok{)}
\NormalTok{Iintra}\OtherTok{\textless{}{-}}\ConstantTok{NULL}
\ControlFlowTok{for}\NormalTok{ (k }\ControlFlowTok{in} \DecValTok{2}\SpecialCharTok{:}\NormalTok{Kmax)\{}
\NormalTok{  resaux}\OtherTok{\textless{}{-}}\FunctionTok{kmeans}\NormalTok{(Datalog[,}\DecValTok{4}\SpecialCharTok{:}\DecValTok{14}\NormalTok{],}\AttributeTok{centers=}\NormalTok{k)}
\NormalTok{  reskmeanscl[,k}\DecValTok{{-}1}\NormalTok{]}\OtherTok{\textless{}{-}}\NormalTok{resaux}\SpecialCharTok{$}\NormalTok{cluster}
\NormalTok{  Iintra}\OtherTok{\textless{}{-}}\FunctionTok{c}\NormalTok{(Iintra,resaux}\SpecialCharTok{$}\NormalTok{tot.withinss)}
\NormalTok{\}}

\NormalTok{df}\OtherTok{\textless{}{-}}\FunctionTok{data.frame}\NormalTok{(}\AttributeTok{K=}\DecValTok{2}\SpecialCharTok{:}\DecValTok{15}\NormalTok{,}\AttributeTok{Iintra=}\NormalTok{Iintra)}
\FunctionTok{ggplot}\NormalTok{(df,}\FunctionTok{aes}\NormalTok{(}\AttributeTok{x=}\NormalTok{K,}\AttributeTok{y=}\NormalTok{Iintra))}\SpecialCharTok{+}\FunctionTok{geom\_line}\NormalTok{()}\SpecialCharTok{+}\FunctionTok{geom\_point}\NormalTok{()}\SpecialCharTok{+}\FunctionTok{xlab}\NormalTok{(}\StringTok{"Nombre de classes"}\NormalTok{)}\SpecialCharTok{+}\FunctionTok{ylab}\NormalTok{(}\StringTok{"Inertie intraclasse"}\NormalTok{)}
\end{Highlighting}
\end{Shaded}

\hypertarget{etude-de-luxe9mission-de-muxe9thane-ancova}{%
\section{Etude de l'émission de méthane
(ANCOVA)}\label{etude-de-luxe9mission-de-muxe9thane-ancova}}

Nous souhaitons étudier l'émission de méthane \(ch4\_t\) en fonction de
l'ammoniac \(nh3\_kg\), du protoxyde d'azote \(n2o\_t\), du type d'EPCI
\(TypeEPCI\) et de l'année \(annee\_inv\). Nous avons des variables
quantitatives et qualitatives, nous allons donc faire une ANCOVA. Nous
effectuons l'ANCOVA sur les données modifiées (par le logarithme).

Dans un premier temps, on considère le modèle complet avec interactions.

\[
(M_1) 
\left\{\begin{array}{l} ch4\_t_{ijk}= \mu + \alpha_i + \beta_j + \gamma_{ij} + (a1 + a2_i + a3_j)\times nh3\_kg_{ijk} + (b1 + b2_i + b3_j)\times n2o\_t_{ijk} + \nu \times n2o\_t_{ijk} \times nh3\_kg_{ijk} +
\varepsilon_{ijk},\ \\
i=1,\ldots,I=6,\ j=1 \ldots,J=2, \ k=1,\ldots,n_{ij}.\\ (\varepsilon_{ijk})_{i,j,k} \textrm{ i.i.d
} \ \mathcal{N}(0,\sigma^2) \end{array}\right. 
\]

\(ch4\_t_{ijk}\) représentes la valeur de la \(k^{ème}\) mesure du
\(ch4\_t\) pour la \(i^{ème}\) année et pour le \(j^{ème}\) type d'EPCI.

\begin{Shaded}
\begin{Highlighting}[]
\NormalTok{complet}\OtherTok{\textless{}{-}}\FunctionTok{lm}\NormalTok{(ch4\_t }\SpecialCharTok{\textasciitilde{}}\NormalTok{ .}\SpecialCharTok{\^{}}\DecValTok{2}\NormalTok{ ,}\AttributeTok{data=}\NormalTok{Datalog\_[}\FunctionTok{c}\NormalTok{(}\DecValTok{3}\NormalTok{,}\DecValTok{10}\NormalTok{,}\DecValTok{12}\NormalTok{, }\DecValTok{14}\NormalTok{,}\DecValTok{15}\NormalTok{)])}
\FunctionTok{summary}\NormalTok{(complet)}
\end{Highlighting}
\end{Shaded}

\begin{verbatim}
## 
## Call:
## lm(formula = ch4_t ~ .^2, data = Datalog_[c(3, 10, 12, 14, 15)])
## 
## Residuals:
##      Min       1Q   Median       3Q      Max 
## -1.54659 -0.23562  0.05262  0.27848  1.80607 
## 
## Coefficients:
##                            Estimate Std. Error t value Pr(>|t|)    
## (Intercept)               -1.395554   1.153374  -1.210 0.226587    
## annee_inv2015              0.391017   1.078546   0.363 0.717028    
## annee_inv2016              0.140902   1.138029   0.124 0.901490    
## annee_inv2017             -0.766577   1.245618  -0.615 0.538424    
## annee_inv2018             -1.526748   1.303239  -1.172 0.241688    
## annee_inv2019             -0.737383   1.275277  -0.578 0.563256    
## nh3_kg                     0.692391   0.126043   5.493 5.06e-08 ***
## n2o_t                     -0.635608   0.192432  -3.303 0.000992 ***
## TypeEPCICC               -11.480929   0.881128 -13.030  < 2e-16 ***
## annee_inv2015:nh3_kg      -0.040240   0.133827  -0.301 0.763718    
## annee_inv2016:nh3_kg      -0.012393   0.141834  -0.087 0.930390    
## annee_inv2017:nh3_kg       0.086343   0.155379   0.556 0.578549    
## annee_inv2018:nh3_kg       0.179818   0.162588   1.106 0.269017    
## annee_inv2019:nh3_kg       0.080817   0.159150   0.508 0.611707    
## annee_inv2015:n2o_t        0.017030   0.149732   0.114 0.909471    
## annee_inv2016:n2o_t       -0.012442   0.159373  -0.078 0.937788    
## annee_inv2017:n2o_t       -0.115455   0.175564  -0.658 0.510937    
## annee_inv2018:n2o_t       -0.216637   0.183242  -1.182 0.237402    
## annee_inv2019:n2o_t       -0.108123   0.179714  -0.602 0.547558    
## annee_inv2015:TypeEPCICC  -0.006751   0.160803  -0.042 0.966520    
## annee_inv2016:TypeEPCICC  -0.046244   0.161614  -0.286 0.774835    
## annee_inv2017:TypeEPCICC  -0.059297   0.160619  -0.369 0.712074    
## annee_inv2018:TypeEPCICC  -0.205710   0.163455  -1.259 0.208513    
## annee_inv2019:TypeEPCICC  -0.149353   0.162379  -0.920 0.357917    
## nh3_kg:n2o_t               0.039984   0.012036   3.322 0.000927 ***
## nh3_kg:TypeEPCICC          1.294907   0.098110  13.198  < 2e-16 ***
## n2o_t:TypeEPCICC          -1.205566   0.107187 -11.247  < 2e-16 ***
## ---
## Signif. codes:  0 '***' 0.001 '**' 0.01 '*' 0.05 '.' 0.1 ' ' 1
## 
## Residual standard error: 0.4432 on 957 degrees of freedom
## Multiple R-squared:  0.8354, Adjusted R-squared:  0.8309 
## F-statistic: 186.8 on 26 and 957 DF,  p-value: < 2.2e-16
\end{verbatim}

Au vue des sorties des tests de student pour chaque coefficient, il
semblerait que nous puissions supprimer certaines de ces interactions.
Nous allons donc essayer de réduire notre modèle. Pour cela, nous
supposons d'abord le modèle suivant sans interactions :

\[
(M_2) 
\left\{\begin{array}{l} ch4\_t_{ijk}= \mu + \alpha_i + \beta_j + \theta \times nh3\_kg_{ijk} + \gamma \times n2o\_t_{ijk} + 
\varepsilon_{ijk},\ \\
i=1,\ldots,I=6,\ j=1 \ldots,J=2, \ k=1,\ldots,n_{ij}.\\ (\varepsilon_{ijk})_{i,j,k} \textrm{ i.i.d
}\mathcal{N}(0,\sigma^2) \end{array}\right. 
\]

Nous effectuons alors un test de sous modèle de Fisher pour voir si nous
pouvons enlever les interactions.

\begin{verbatim}
## Analysis of Variance Table
## 
## Model 1: ch4_t ~ annee_inv + nh3_kg + n2o_t + TypeEPCI
## Model 2: ch4_t ~ (annee_inv + nh3_kg + n2o_t + TypeEPCI)^2
##   Res.Df    RSS Df Sum of Sq      F    Pr(>F)    
## 1    975 226.33                                  
## 2    957 187.95 18    38.381 10.857 < 2.2e-16 ***
## ---
## Signif. codes:  0 '***' 0.001 '**' 0.01 '*' 0.05 '.' 0.1 ' ' 1
\end{verbatim}

Notre p-valeur est très faible, nous ne pouvons pas supprimer toutes les
interactions. Nous allons à la place utiliser un algorithme de sélection
de variable pour réduire notre modèle.

Nous choissisons la méthode backward avec les critères BIC et AIC.

\begin{verbatim}
## 
## Call:
## lm(formula = ch4_t ~ annee_inv + nh3_kg + n2o_t + TypeEPCI + 
##     nh3_kg:n2o_t + nh3_kg:TypeEPCI + n2o_t:TypeEPCI, data = Datalog_[c(3, 
##     10, 12, 14, 15)])
## 
## Coefficients:
##       (Intercept)      annee_inv2015      annee_inv2016      annee_inv2017  
##          -1.58579           -0.03908           -0.09280           -0.18379  
##     annee_inv2018      annee_inv2019             nh3_kg              n2o_t  
##          -0.29054           -0.27095            0.72586           -0.67216  
##        TypeEPCICC       nh3_kg:n2o_t  nh3_kg:TypeEPCICC   n2o_t:TypeEPCICC  
##         -11.54616            0.03941            1.29888           -1.22017
\end{verbatim}

\begin{verbatim}
## 
## Call:
## lm(formula = ch4_t ~ annee_inv + nh3_kg + n2o_t + TypeEPCI + 
##     nh3_kg:n2o_t + nh3_kg:TypeEPCI + n2o_t:TypeEPCI, data = Datalog_[c(3, 
##     10, 12, 14, 15)])
## 
## Coefficients:
##       (Intercept)      annee_inv2015      annee_inv2016      annee_inv2017  
##          -1.58579           -0.03908           -0.09280           -0.18379  
##     annee_inv2018      annee_inv2019             nh3_kg              n2o_t  
##          -0.29054           -0.27095            0.72586           -0.67216  
##        TypeEPCICC       nh3_kg:n2o_t  nh3_kg:TypeEPCICC   n2o_t:TypeEPCICC  
##         -11.54616            0.03941            1.29888           -1.22017
\end{verbatim}

Les deux algorithmes nous donne le même sous modèle :

\[
(M_3) 
\left\{\begin{array}{l} ch4\_t_{ijk}= \mu + \alpha_i + \beta_j + (a1 + a3_j)\times nh3\_kg_{ijk} + (b1 + b3_j)\times n2o\_t_{ijk} + \nu \times n2o\_t_{ijk} \times nh3\_kg_{ijk} +
\varepsilon_{ijk},\ \\
i=1,\ldots,I=6,\ j=1 \ldots,J=2, \ k=1,\ldots,n_{ij}.\\ (\varepsilon_{ijk})_{i,j,k} \textrm{ i.i.d
} \ \mathcal{N}(0,\sigma^2) \end{array}\right. 
\]

Nous avons supprimé 3 coefficients. Nous vérifions ce resultat avec un
test de sous modèle.

\begin{Shaded}
\begin{Highlighting}[]
\NormalTok{M3 }\OtherTok{=} \FunctionTok{lm}\NormalTok{(ch4\_t }\SpecialCharTok{\textasciitilde{}}\NormalTok{ annee\_inv }\SpecialCharTok{+}\NormalTok{ nh3\_kg }\SpecialCharTok{+}\NormalTok{ n2o\_t }\SpecialCharTok{+}\NormalTok{ TypeEPCI }\SpecialCharTok{+} 
\NormalTok{    nh3\_kg}\SpecialCharTok{:}\NormalTok{n2o\_t }\SpecialCharTok{+}\NormalTok{ nh3\_kg}\SpecialCharTok{:}\NormalTok{TypeEPCI }\SpecialCharTok{+}\NormalTok{ n2o\_t}\SpecialCharTok{:}\NormalTok{TypeEPCI, }\AttributeTok{data =}\NormalTok{ Datalog\_[}\FunctionTok{c}\NormalTok{(}\DecValTok{3}\NormalTok{, }
    \DecValTok{10}\NormalTok{, }\DecValTok{12}\NormalTok{, }\DecValTok{14}\NormalTok{, }\DecValTok{15}\NormalTok{)])}
\end{Highlighting}
\end{Shaded}

\begin{verbatim}
## Analysis of Variance Table
## 
## Model 1: ch4_t ~ annee_inv + nh3_kg + n2o_t + TypeEPCI + nh3_kg:n2o_t + 
##     nh3_kg:TypeEPCI + n2o_t:TypeEPCI
## Model 2: ch4_t ~ (annee_inv + nh3_kg + n2o_t + TypeEPCI)^2
##   Res.Df    RSS Df Sum of Sq      F Pr(>F)
## 1    972 188.74                           
## 2    957 187.95 15   0.78493 0.2664 0.9977
\end{verbatim}

Notre p-valeur est bien supérieur à \(0.05\) donc on ne rejette pas ce
sous modèle.

Par curiosité, nous avons voulu supprimer certaines interactions
manuellement. Nous avons trouvé que nous pouvions supprimer
l'interaction entre \(nh3\_kg\) et \(n2o\_t\) et toujours avoir une
p-valeur supérieure à \(0.05\).

\begin{verbatim}
## Analysis of Variance Table
## 
## Model 1: ch4_t ~ annee_inv + nh3_kg + n2o_t + TypeEPCI + nh3_kg:TypeEPCI + 
##     n2o_t:TypeEPCI
## Model 2: ch4_t ~ (annee_inv + nh3_kg + n2o_t + TypeEPCI)^2
##   Res.Df    RSS Df Sum of Sq      F Pr(>F)
## 1    973 190.85                           
## 2    957 187.95 16    2.8963 0.9217 0.5437
\end{verbatim}

Nous pourrions alors considérer le modèle final suivant :

\[
(M_4) 
\left\{\begin{array}{l} ch4\_t_{ijk}= \mu + \alpha_i + \beta_j + (a1 + a3_j)\times nh3\_kg_{ijk} + (b1 + b3_j)\times n2o\_t_{ijk} +
\varepsilon_{ijk},\ \\
i=1,\ldots,I=6,\ j=1 \ldots,J=2, \ k=1,\ldots,n_{ij}.\\ (\varepsilon_{ijk})_{i,j,k} \textrm{ i.i.d
} \ \mathcal{N}(0,\sigma^2) \end{array}\right. 
\]

\hypertarget{duxe9passement-duxe9mission-de-muxe9thane-moduxe8le-linuxe9aire-guxe9nuxe9ralisuxe9}{%
\subsection{Dépassement d'émission de méthane (Modèle linéaire
généralisé)}\label{duxe9passement-duxe9mission-de-muxe9thane-moduxe8le-linuxe9aire-guxe9nuxe9ralisuxe9}}

Dans cette section, nous allons expliquer le dépassement d'émission de
méthane (\(ch4\))de \(1000t\) par an en fonction de l'ammoniac
(\(nh3\)), du protoxyde d'azote (\(n2o\)), du type d'EPCI et de l'année
par un modèle linéaire généralisé.

Nous créons la nouvelle variable booléenne \(Data\_mlg\), valant 1 si le
taux de méthane \(ch4\_t\) est supérieur à \(1000t\), 0 sinon.

\begin{verbatim}
##   annee_inv   nh3_kg    n2o_t        TypeEPCI dep_met_1000
## 1      2019 11.80325 2.839897              CC        FALSE
## 2      2019 11.64862 2.983407              CC        FALSE
## 3      2019 11.17293 2.868240              CC        FALSE
## 4      2019 12.40602 3.931355              CC        FALSE
## 5      2019 11.77857 3.204371 CA/CU/Métropole        FALSE
## 6      2019 11.61738 4.090337 CA/CU/Métropole        FALSE
\end{verbatim}

La variable \(dep\_met\_1000\) étant binaire, nous allons utiliser la
régression logistique.

Dans un premier temps, nous créons le modèle complet avec interactions,
modèle que l'on note \((\mathcal{M}_{GL_{1}})\) :

\[
\begin{equation*}
(\text{M}_{GL_{1}}) : 
\begin{cases}
\begin{aligned}
\text{dep_met_1000}_i \sim & ~ \mathcal{B}(\pi(x_i)) \text{, dep_met_1000}_1 \text{, ..., dep_met_1000}_n \text{ indépendant.}& \\ 
\\
\text{logit}[\pi(x_i)] = \ln(\cfrac{\pi(x_i)}{1 - \pi(x_i)})  = & ~ \mu  +  \theta_1\cdot\text{nh3_kg}_i  + \theta_2\cdot\text{n2o_t}_i  +  \gamma\cdot\text{nh3_kg}_i\cdot\text{n2o_t}_i  \\ 
& + (\beta_1  +  \beta_2\cdot\text{nh3_kg}_i ~ + \beta_3\cdot\text{n2o_t}_i )\mathbb{1}_{\text{TypeEPCI}_i = \text{CC}} \\
&  + \sum_{k=1}^5 (\delta_{1k}  +  \delta_{2k}\cdot\text{nh3_kg}_i ~ + \delta_{3k}\cdot\text{n2o_t}_i )\mathbb{1}_{\text{annee}_i = 2014 + k} \\
& + \sum_{k=1}^5 (\kappa_{1k} \cdot \mathbb{1}_{\text{TypeEPCI}_i = \text{CC}}\cdot\mathbb{1}_{\text{annee}_i = 2014 + k}) \\
\end{aligned}
\end{cases}
\end{equation*}
\]

\begin{verbatim}
## Warning: glm.fit: des probabilités ont été ajustées numériquement à 0 ou 1
\end{verbatim}

Nous observons que certaines variables du modèle complet ont une
p-valeur \textgreater{} 0.05. Nous allons nous intéresser à la
simplification du modèle. Nous commencons par faire un test de
sous-modèle entre le modèle complet et le modèle sans interactions.

Nous obtenons une p-valeur \textless\textless{} 0.05. On rejette donc
l'hypothèse \(\mathcal{H_0}\), et nous devons conserver dans un premier
temps le modèle complet, avec toutes les interactions. Nous allons
maintenant appliquer des algorithmes de selection de variables pour
réduire le modèle complet.

\begin{Shaded}
\begin{Highlighting}[]
\NormalTok{step.backward }\OtherTok{\textless{}{-}} \FunctionTok{step}\NormalTok{(mlg\_complet)}
\end{Highlighting}
\end{Shaded}

\begin{verbatim}
## Start:  AIC=459.14
## dep_met_1000 ~ (annee_inv + nh3_kg + n2o_t + TypeEPCI)^2
\end{verbatim}

\begin{verbatim}
## Warning: glm.fit: des probabilités ont été ajustées numériquement à 0 ou 1

## Warning: glm.fit: des probabilités ont été ajustées numériquement à 0 ou 1

## Warning: glm.fit: des probabilités ont été ajustées numériquement à 0 ou 1

## Warning: glm.fit: des probabilités ont été ajustées numériquement à 0 ou 1

## Warning: glm.fit: des probabilités ont été ajustées numériquement à 0 ou 1

## Warning: glm.fit: des probabilités ont été ajustées numériquement à 0 ou 1
\end{verbatim}

\begin{verbatim}
##                      Df Deviance    AIC
## - annee_inv:TypeEPCI  5   410.45 454.45
## - annee_inv:n2o_t     5   411.49 455.49
## - annee_inv:nh3_kg    5   411.53 455.53
## <none>                    405.14 459.14
## - nh3_kg:n2o_t        1   415.64 467.64
## - nh3_kg:TypeEPCI     1   448.45 500.45
## - n2o_t:TypeEPCI      1   449.23 501.23
\end{verbatim}

\begin{verbatim}
## Warning: glm.fit: des probabilités ont été ajustées numériquement à 0 ou 1
\end{verbatim}

\begin{verbatim}
## 
## Step:  AIC=454.45
## dep_met_1000 ~ annee_inv + nh3_kg + n2o_t + TypeEPCI + annee_inv:nh3_kg + 
##     annee_inv:n2o_t + nh3_kg:n2o_t + nh3_kg:TypeEPCI + n2o_t:TypeEPCI
\end{verbatim}

\begin{verbatim}
## Warning: glm.fit: des probabilités ont été ajustées numériquement à 0 ou 1

## Warning: glm.fit: des probabilités ont été ajustées numériquement à 0 ou 1

## Warning: glm.fit: des probabilités ont été ajustées numériquement à 0 ou 1

## Warning: glm.fit: des probabilités ont été ajustées numériquement à 0 ou 1

## Warning: glm.fit: des probabilités ont été ajustées numériquement à 0 ou 1
\end{verbatim}

\begin{verbatim}
##                    Df Deviance    AIC
## - annee_inv:n2o_t   5   416.27 450.27
## - annee_inv:nh3_kg  5   417.18 451.18
## <none>                  410.45 454.45
## - nh3_kg:n2o_t      1   421.51 463.51
## - nh3_kg:TypeEPCI   1   450.01 492.01
## - n2o_t:TypeEPCI    1   451.12 493.12
\end{verbatim}

\begin{verbatim}
## Warning: glm.fit: des probabilités ont été ajustées numériquement à 0 ou 1
\end{verbatim}

\begin{verbatim}
## 
## Step:  AIC=450.27
## dep_met_1000 ~ annee_inv + nh3_kg + n2o_t + TypeEPCI + annee_inv:nh3_kg + 
##     nh3_kg:n2o_t + nh3_kg:TypeEPCI + n2o_t:TypeEPCI
\end{verbatim}

\begin{verbatim}
## Warning: glm.fit: des probabilités ont été ajustées numériquement à 0 ou 1

## Warning: glm.fit: des probabilités ont été ajustées numériquement à 0 ou 1

## Warning: glm.fit: des probabilités ont été ajustées numériquement à 0 ou 1
\end{verbatim}

\begin{verbatim}
##                    Df Deviance    AIC
## - annee_inv:nh3_kg  5   418.78 442.78
## <none>                  416.27 450.27
## - nh3_kg:n2o_t      1   427.24 459.24
## - nh3_kg:TypeEPCI   1   462.59 494.59
## - n2o_t:TypeEPCI    1   463.99 495.99
\end{verbatim}

\begin{verbatim}
## Warning: glm.fit: des probabilités ont été ajustées numériquement à 0 ou 1
\end{verbatim}

\begin{verbatim}
## 
## Step:  AIC=442.78
## dep_met_1000 ~ annee_inv + nh3_kg + n2o_t + TypeEPCI + nh3_kg:n2o_t + 
##     nh3_kg:TypeEPCI + n2o_t:TypeEPCI
\end{verbatim}

\begin{verbatim}
## Warning: glm.fit: des probabilités ont été ajustées numériquement à 0 ou 1

## Warning: glm.fit: des probabilités ont été ajustées numériquement à 0 ou 1

## Warning: glm.fit: des probabilités ont été ajustées numériquement à 0 ou 1
\end{verbatim}

\begin{verbatim}
##                   Df Deviance    AIC
## <none>                 418.78 442.78
## - nh3_kg:n2o_t     1   431.39 453.39
## - annee_inv        5   446.78 460.78
## - nh3_kg:TypeEPCI  1   464.51 486.51
## - n2o_t:TypeEPCI   1   466.07 488.07
\end{verbatim}

\begin{verbatim}
## Warning: glm.fit: des probabilités ont été ajustées numériquement à 0 ou 1

## Warning: glm.fit: des probabilités ont été ajustées numériquement à 0 ou 1

## Warning: glm.fit: des probabilités ont été ajustées numériquement à 0 ou 1

## Warning: glm.fit: des probabilités ont été ajustées numériquement à 0 ou 1

## Warning: glm.fit: des probabilités ont été ajustées numériquement à 0 ou 1

## Warning: glm.fit: des probabilités ont été ajustées numériquement à 0 ou 1

## Warning: glm.fit: des probabilités ont été ajustées numériquement à 0 ou 1

## Warning: glm.fit: des probabilités ont été ajustées numériquement à 0 ou 1

## Warning: glm.fit: des probabilités ont été ajustées numériquement à 0 ou 1

## Warning: glm.fit: des probabilités ont été ajustées numériquement à 0 ou 1

## Warning: glm.fit: des probabilités ont été ajustées numériquement à 0 ou 1

## Warning: glm.fit: des probabilités ont été ajustées numériquement à 0 ou 1

## Warning: glm.fit: des probabilités ont été ajustées numériquement à 0 ou 1

## Warning: glm.fit: des probabilités ont été ajustées numériquement à 0 ou 1

## Warning: glm.fit: des probabilités ont été ajustées numériquement à 0 ou 1

## Warning: glm.fit: des probabilités ont été ajustées numériquement à 0 ou 1

## Warning: glm.fit: des probabilités ont été ajustées numériquement à 0 ou 1

## Warning: glm.fit: des probabilités ont été ajustées numériquement à 0 ou 1

## Warning: glm.fit: des probabilités ont été ajustées numériquement à 0 ou 1

## Warning: glm.fit: des probabilités ont été ajustées numériquement à 0 ou 1
\end{verbatim}

\begin{verbatim}
## 
## Call:  glm(formula = dep_met_1000 ~ annee_inv + nh3_kg + n2o_t + TypeEPCI + 
##     nh3_kg:n2o_t + nh3_kg:TypeEPCI + n2o_t:TypeEPCI, family = binomial(link = "logit"), 
##     data = Data_mlg)
## 
## Coefficients:
##       (Intercept)      annee_inv2015      annee_inv2016      annee_inv2017  
##         -121.7763            -0.4846            -0.6902            -1.1280  
##     annee_inv2018      annee_inv2019             nh3_kg              n2o_t  
##           -1.8518            -1.8314            10.0779            16.4354  
##        TypeEPCICC       nh3_kg:n2o_t  nh3_kg:TypeEPCICC   n2o_t:TypeEPCICC  
##         -117.6891            -1.3993            13.1830           -12.2932  
## 
## Degrees of Freedom: 971 Total (i.e. Null);  960 Residual
## Null Deviance:       1193 
## Residual Deviance: 418.8     AIC: 442.8
\end{verbatim}

\begin{verbatim}
## Start:  AIC=459.14
## dep_met_1000 ~ (annee_inv + nh3_kg + n2o_t + TypeEPCI)^2
\end{verbatim}

\begin{verbatim}
## Warning: glm.fit: des probabilités ont été ajustées numériquement à 0 ou 1

## Warning: glm.fit: des probabilités ont été ajustées numériquement à 0 ou 1

## Warning: glm.fit: des probabilités ont été ajustées numériquement à 0 ou 1

## Warning: glm.fit: des probabilités ont été ajustées numériquement à 0 ou 1

## Warning: glm.fit: des probabilités ont été ajustées numériquement à 0 ou 1

## Warning: glm.fit: des probabilités ont été ajustées numériquement à 0 ou 1
\end{verbatim}

\begin{verbatim}
##                      Df Deviance    AIC
## - annee_inv:TypeEPCI  5   410.45 454.45
## - annee_inv:n2o_t     5   411.49 455.49
## - annee_inv:nh3_kg    5   411.53 455.53
## <none>                    405.14 459.14
## - nh3_kg:n2o_t        1   415.64 467.64
## - nh3_kg:TypeEPCI     1   448.45 500.45
## - n2o_t:TypeEPCI      1   449.23 501.23
\end{verbatim}

\begin{verbatim}
## Warning: glm.fit: des probabilités ont été ajustées numériquement à 0 ou 1
\end{verbatim}

\begin{verbatim}
## 
## Step:  AIC=454.45
## dep_met_1000 ~ annee_inv + nh3_kg + n2o_t + TypeEPCI + annee_inv:nh3_kg + 
##     annee_inv:n2o_t + nh3_kg:n2o_t + nh3_kg:TypeEPCI + n2o_t:TypeEPCI
\end{verbatim}

\begin{verbatim}
## Warning: glm.fit: des probabilités ont été ajustées numériquement à 0 ou 1

## Warning: glm.fit: des probabilités ont été ajustées numériquement à 0 ou 1

## Warning: glm.fit: des probabilités ont été ajustées numériquement à 0 ou 1

## Warning: glm.fit: des probabilités ont été ajustées numériquement à 0 ou 1

## Warning: glm.fit: des probabilités ont été ajustées numériquement à 0 ou 1
\end{verbatim}

\begin{verbatim}
##                    Df Deviance    AIC
## - annee_inv:n2o_t   5   416.27 450.27
## - annee_inv:nh3_kg  5   417.18 451.18
## <none>                  410.45 454.45
## - nh3_kg:n2o_t      1   421.51 463.51
## - nh3_kg:TypeEPCI   1   450.01 492.01
## - n2o_t:TypeEPCI    1   451.12 493.12
\end{verbatim}

\begin{verbatim}
## Warning: glm.fit: des probabilités ont été ajustées numériquement à 0 ou 1
\end{verbatim}

\begin{verbatim}
## 
## Step:  AIC=450.27
## dep_met_1000 ~ annee_inv + nh3_kg + n2o_t + TypeEPCI + annee_inv:nh3_kg + 
##     nh3_kg:n2o_t + nh3_kg:TypeEPCI + n2o_t:TypeEPCI
\end{verbatim}

\begin{verbatim}
## Warning: glm.fit: des probabilités ont été ajustées numériquement à 0 ou 1

## Warning: glm.fit: des probabilités ont été ajustées numériquement à 0 ou 1

## Warning: glm.fit: des probabilités ont été ajustées numériquement à 0 ou 1
\end{verbatim}

\begin{verbatim}
##                    Df Deviance    AIC
## - annee_inv:nh3_kg  5   418.78 442.78
## <none>                  416.27 450.27
## - nh3_kg:n2o_t      1   427.24 459.24
## - nh3_kg:TypeEPCI   1   462.59 494.59
## - n2o_t:TypeEPCI    1   463.99 495.99
\end{verbatim}

\begin{verbatim}
## Warning: glm.fit: des probabilités ont été ajustées numériquement à 0 ou 1
\end{verbatim}

\begin{verbatim}
## 
## Step:  AIC=442.78
## dep_met_1000 ~ annee_inv + nh3_kg + n2o_t + TypeEPCI + nh3_kg:n2o_t + 
##     nh3_kg:TypeEPCI + n2o_t:TypeEPCI
\end{verbatim}

\begin{verbatim}
## Warning: glm.fit: des probabilités ont été ajustées numériquement à 0 ou 1

## Warning: glm.fit: des probabilités ont été ajustées numériquement à 0 ou 1

## Warning: glm.fit: des probabilités ont été ajustées numériquement à 0 ou 1
\end{verbatim}

\begin{verbatim}
##                   Df Deviance    AIC
## <none>                 418.78 442.78
## - nh3_kg:n2o_t     1   431.39 453.39
## - annee_inv        5   446.78 460.78
## - nh3_kg:TypeEPCI  1   464.51 486.51
## - n2o_t:TypeEPCI   1   466.07 488.07
\end{verbatim}

\begin{verbatim}
## 
## Call:  glm(formula = dep_met_1000 ~ annee_inv + nh3_kg + n2o_t + TypeEPCI + 
##     nh3_kg:n2o_t + nh3_kg:TypeEPCI + n2o_t:TypeEPCI, family = binomial(link = "logit"), 
##     data = Data_mlg)
## 
## Coefficients:
##       (Intercept)      annee_inv2015      annee_inv2016      annee_inv2017  
##         -121.7763            -0.4846            -0.6902            -1.1280  
##     annee_inv2018      annee_inv2019             nh3_kg              n2o_t  
##           -1.8518            -1.8314            10.0779            16.4354  
##        TypeEPCICC       nh3_kg:n2o_t  nh3_kg:TypeEPCICC   n2o_t:TypeEPCICC  
##         -117.6891            -1.3993            13.1830           -12.2932  
## 
## Degrees of Freedom: 971 Total (i.e. Null);  960 Residual
## Null Deviance:       1193 
## Residual Deviance: 418.8     AIC: 442.8
\end{verbatim}

Nous utilisons 3 méthodes d'algorithme de selection de variable en
méthode backward et les 3 s'accordent sur le même modèle.

\[
\begin{equation*}
(\text{M}_{GL_{2}}) : 
\begin{cases}
\begin{aligned}
\text{dep_met_1000}_i \sim & ~ \mathcal{B}(\pi(x_i)) \text{, dep_met_1000}_1 \text{, ..., dep_met_1000}_n \text{ indépendant.}& \\ 
\\
\text{logit}[\pi(x_i)] = \ln(\cfrac{\pi(x_i)}{1 - \pi(x_i)})  = & ~ \mu  +  \theta_1\cdot\text{nh3_kg}_i  + \theta_2\cdot\text{n2o_t}_i  +  \gamma\cdot\text{nh3_kg}_i\cdot\text{n2o_t}_i  \\ 
& + (\beta_1  +  \beta_2\cdot\text{nh3_kg}_i ~ + \beta_3\cdot\text{n2o_t}_i )\mathbb{1}_{\text{TypeEPCI}_i = \text{CC}} \\
\end{aligned}
\end{cases}
\end{equation*}
\]

\begin{verbatim}
## Warning: glm.fit: des probabilités ont été ajustées numériquement à 0 ou 1
\end{verbatim}

Nous effectuons un test de sous-modèle pour le valider. La p-valeur est
de 0.5528 \textgreater{} 0.05. Nous ne rejetons alors pas le modèle
réduit.

\begin{verbatim}
## Analysis of Deviance Table
## 
## Model 1: dep_met_1000 ~ annee_inv + nh3_kg + n2o_t + TypeEPCI + nh3_kg:n2o_t + 
##     nh3_kg:TypeEPCI + n2o_t:TypeEPCI
## Model 2: dep_met_1000 ~ (annee_inv + nh3_kg + n2o_t + TypeEPCI)^2
##   Resid. Df Resid. Dev Df Deviance Pr(>Chi)
## 1       960     418.78                     
## 2       945     405.14 15   13.643   0.5528
\end{verbatim}

On a :
\(T := \mathcal{D}(M_0) - \mathcal{D}(M_1) \sim \chi^2(k_1 - k_0)\),
avec
\(\mathcal{D}(M) = -2\{l(Y,\hat{\theta}) - l(Y,\hat{\theta}_{sat})\}\).
Avec les sorties R, on constate : \(T^{obs} = 13,643,\) puis pvaleur =
\(\mathbb{P}(T > T^{obs}) = 0.55\).

On a pvaleur \(> 0.05\), on conserve donc le modèle réduit au risque
\(5\%\).

\begin{verbatim}
## [1] 0.6490327
\end{verbatim}

Le \(\text{pseudoR}^2\) du modèle réduit vaut 0.6490327, c'est une
valeur satisfaisante pour fitter notre modèle.

\end{document}
